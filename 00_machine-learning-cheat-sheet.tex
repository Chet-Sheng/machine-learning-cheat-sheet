%%%%%%%%%%%%%%%%%%%% book.tex %%%%%%%%%%%%%%%%%%%%%%%%%%%%%
%
% sample root file for the chapters of your "monograph"
%
% Use this file as a template for your own input.
%
%%%%%%%%%%%%%%%% Springer-Verlag %%%%%%%%%%%%%%%%%%%%%%%%%%


% RECOMMENDED %%%%%%%%%%%%%%%%%%%%%%%%%%%%%%%%%%%%%%%%%%%%%%%%%%%
%\documentclass[graybox, envcountchap, twocolumn]{styles02/svmono}
\documentclass[norunningheads, graybox, envcountchap]{styles02/svmono}
\usepackage[a4paper, 
left=20mm,
right=20mm,
top=20mm,
bottom=20mm,
%headheight=50mm
footskip=5mm
]{geometry}

%\usepackage{bookmark}
% set indent to be 0
\setlength{\parindent}{0pt}
%==========================================
% chapter title formating
\usepackage{titlesec}
\titleformat{\chapter}[display]
{\normalfont\Huge \scshape}{\titlerule}{-28pt}{\thechapter\ \ \Huge}[\vspace{-2pt}\titlerule]
\titleformat{name=\chapter,numberless}[display]
{\normalfont\Huge \scshape}{\titlerule}{-28pt}{\Huge}[\vspace{-2pt}\titlerule]
\titlespacing*{\chapter}{0pt}{0pt}{20pt}

%==========================================
% head and foot formating
\usepackage{fancyhdr}
\pagestyle{fancy}
%\fancyhf{}
\setlength\headheight{15pt} 
%\fancyfoot[R]{\thepage}
%==========================================

%\textwidth of 117 mm or 27-3/4 pi pi 
%a \textheight of 191 mm or 45-1/6 pi 
%a \headsep of 12 pt (space between the running head and text).

% choose options for [] as required from the list
% in the Reference Guide

%==========================================
% Adding Parts
%
\usepackage{tocloft}

\makeatletter
\@addtoreset{section}{part}
\makeatother
\newlength\mylen
\renewcommand\thepart{\arabic{part}}
\renewcommand\cftpartpresnum{Part~}
\settowidth\mylen{\bfseries\cftpartpresnum\cftpartaftersnum}
\addtolength\cftpartnumwidth{\mylen}

%==========================================
%%%%%%%%%%%%%%%%%%%%%%%%%%%%%%%%%%%%%%%%%%%%%%%%%%%%%%%%%%%%%%%%%
% graybox example
%\begin{svgraybox}...\end{svgraybox
%%%%%%%%%%%%%%%%%%%%%%%%%%%%%%%%%%%%%%%%%%%%%%%%%%%%%%%%%%%%%%%%%

%
\usepackage{type1cm}         

\usepackage{makeidx}         % allows index generation
\usepackage{graphicx}        % standard LaTeX graphics tool
% when including figure files
\usepackage{multicol}        % used for the two-column index
\usepackage[bottom]{footmisc}% places footnotes at page bottom

% see the list of further useful packages
% in the Reference Guide

% please use the style svind.ist with
% your makeindex program

%%%%%%%%%%%%%%%%%%%%%%%%%%%%%%%%%%%%%%%%%%%%%%%%%%%%%%%%%%%%%%%%%%%%%

\usepackage{amssymb,amsmath,bm}
\DeclareMathAlphabet{\mathcal}{OMS}{cmsy}{m}{n}
\usepackage{textcomp}
\newcommand\abs[1]{\left\lvert#1\right\rvert}
\usepackage{longtable}
\usepackage{algorithm2e}
\usepackage{tocbibind}
\usepackage[toc]{multitoc}
\renewcommand{\bibname}{References}

%%----------------------------------------------------------------------------
\usepackage{mathptmx}        % selects Times Roman as basic font
\usepackage{helvet}          % selects Helvetica as sans-serif font
\usepackage{courier}         % selects Courier as typewriter font
%\usepackage{type1cm}        % activate if the above 3 fonts are 
% not available on your system

% when including figure files
\usepackage[justification=centering]{caption}
\usepackage{subfig}
\usepackage{multirow}
\usepackage[
bookmarksnumbered=true,
bookmarksopen=true,
%bookmarks=true
colorlinks=true,
linkcolor=blue,
anchorcolor=blue,
citecolor=blue
]{hyperref}

\graphicspath{{figures/}}

% see the list of further useful packages in the Reference Guide

\makeindex             % used for the subject index
% please use the style svind.ist with
% your makeindex program

\begin{document}
	
%	\author{soulmachine@gmail.com}
%	\title{Machine Learning Cheat Sheet}
%	\subtitle{Classical equations, diagrams and tricks in machine learning}
	\author{Huafeng Sheng}
	\title{Machine Learning Notebook}
	\subtitle{-- Personalized Edition}
	\maketitle
	
	\frontmatter%%%%%%%%%%%%%%%%%%%%%%%%%%%%%%%%%%%%%%%%%%%%%%%%%%%%%%
	
%	\include{titlepage}
	%\include{dedic}
	%\include{foreword}
	\include{preface}
	%\include{acknow}
	
	\tableofcontents
	%\include{acronym}
	\include{notation}
	
	
	\mainmatter%%%%%%%%%%%%%%%%%%%%%%%%%%%%%%%%%%%%%%%%%%%%%%%%%%%%%%%
	\part{Supervised Learning and Unsupervised Learning}
	\include{chapterIntroduction}
	\include{chapterProbability}
	\include{chapterGenerativeModels}
	\include{chapterMVN}
	\include{chapterBayesianStatistics}
	\include{chapterFrequentistStatistics}
	\include{chapterLinearRegression}
	\include{chapterLogisticRegression}
	\include{chapterGLM}
	\include{chapterDGM}
	\include{chapterEM}
	\include{chapterLatentLinearModels}
	\include{chapterSparseLinearModels}
	\include{chapterKernels}
	\include{chapterGP}
	\include{chapterABM}
	\include{chapterHMM}
	\include{chapterSSM}
	\include{chapterUGM}
	\include{chapterExactInferenceForGraphicalModels}
	\include{chapterVariationalInference}
	\include{chapterMoreVariationalInference}
	\include{chapterMonteCarloInference}
	\include{chapterMCMC}
	\include{chapterClustering}
	\include{chapterStructureLearning}
	\include{chapterLVM}
	\part{Deep Learning}
	\chapter{Neural Network Fundations}
\section{Basic single layer networks}

\subsection{Linear Layer}
It is known as 'fully connected'/ 'dense' layers

\begin{align}
	y=\sum_{i}w_ix_i+b
\end{align}

	\includegraphics[width=0.4\textwidth]{figures/linear01}

\begin{align}
	\textbf{y}= \textbf{W x} + \textbf{b}
\end{align}	


	\includegraphics[width=0.4\textwidth]{figures/linear02}
	



\subsection{Sigmoid Layer}
Sigmoid layer can be used as binary-class classifier with 2-way gating.\\
\textbf{Activation function:}

\begin{align}
	\sigma(x)=\frac{1}{1+ e^{-x}} = \frac{e^x}{1+e^{x}}
\end{align}

\begin{center}
	\includegraphics[width=0.45\textwidth]{figures/Sigmoid}
\end{center}


The logistic sigmoid function $\sigma_\beta=1/1+e^{-\beta x} $. The parameter $ \beta $ determines the steepness of the sigmoid. The full (blue) line is for $ \beta=1 $ and the dashed (red) for $ \beta=10 $. As $ \beta \rightarrow \infty $, the logistic sigmoid tends to a Heaviside step function. \\


\textbf{Linear Layer + Sigmoid Layer as a \underline{binary classifier}\\ (Logistirc Regression):}

\begin{align}
	y = f(x) = \sigma(y(x)) = \frac{1}{1+e^{-y}}= \frac{1}{1+e^{-(Wx+b)}}
\end{align}

\begin{center}
	\includegraphics[width=0.55\textwidth]{figures/sigmoid02}
\end{center} 

\subsection{Softmax Layer}
Softmax layer can be used as multi-class classification and also in soft/differentiably multi-way gating/routing.

\begin{flalign}
	&&y &= softmax(x)&\\
	&\text{with, }&y_i &=\dfrac{e^{x_i}}{\sum_{j=1}^{K}e^{x_j}}&
\end{flalign}


\textbf{LinearLayer+SoftmaxLayer as a \underline{multi-class classifier}:}\\

\vspace{-10pt}

\begin{align}
	 y_i=\dfrac{e^{\sum_{j}w_{ij}x_i+b_i}}{\sum_{k}e^{\sum_{j}w_{kj}x_j+b_k}} 
\end{align}


\subsection{Rectified-linear Layer (ReLU)}
ReLU is simpler/ cheaper than sigmoid, and it is prevalent in modern neural nets.
\begin{flalign}
	&&y&=ReLU(x) &\\ 
	&\text{with, }&y_i &=max(0, x_i)&
\end{flalign}
%\vspace{-10pt}
\begin{center}
		\includegraphics[width=0.35\textwidth]{figures/ReLU}
\end{center}








	\include{chapterOptimization}
	%
	
	\backmatter%%%%%%%%%%%%%%%%%%%%%%%%%%%%%%%%%%%%%%%%%%%%%%%%%%%%%%%
	\appendix
	\include{glossary}
	\printindex
	
	%%%%%%%%%%%%%%%%%%%%%%%%%%%%%%%%%%%%%%%%%%%%%%%%%%%%%%%%%%%%%%%%%%%%%%
	
\end{document}

